\documentclass[twoside,a4paper,12pt,english,draft]{anstrans}

%%%% packages and definitions (optional)
\usepackage{graphicx} % allows inclusion of graphics
\usepackage{booktabs} % nice rules (thick lines) for tables
\usepackage{microtype} % improves typography for PDF


\newcommand{\SN}{S$_N$}
\renewcommand{\vec}[1]{\bm{#1}} % vector is bold italic
\newcommand{\vd}{\bm{\cdot}} % slightly bold vector dot
\newcommand{\grad}{\vec{\nabla}} % gradient
\newcommand{\ud}{\mathop{}\!\mathrm{d}} % upright derivative symbol


%%%% changes from the original 'anstrans' class
\usepackage{fancyhdr} % allows headers and footers

\renewcommand\headrule{} % remove underline in the header
\setcounter{secnumdepth}{1}
\renewcommand{\thesection}{\Roman{section}.}
\renewcommand{\thesubsection}{\arabic{subsection}.}
\renewcommand{\thesubsubsection}{\Alph{subsubsection}.}
\makeatletter
\renewcommand*{\@seccntformat}[1]{\csname the#1\endcsname\hspace{1mm}}
\makeatother


%%%% Header
%\pagestyle{fancy}
%\fancyhf{}
%\fancyhead[L]{\fontsize{9}{9} \itshape
%M\&C 2017 - International Conference on Mathematics \& Computational Methods Applied to Nuclear Science \& Engineering,
%\\ Jeju, Korea, April 16-20, 2017, on USB (2017)
%}


%%%% Maketitle
\title{Serpent nuclear code OpenCL extension OR Memory management in Serpent Nuclear code
for developers}
\author{Vitor Vasconcelos$^{*}$}

\institute{
  $^{*}$Centro de Desenvolvimento da Tecnologia Nuclear - CDTN, Comiss\~ao Nacional de Energia Nuclear - CNEN,
  Av. Presidente Ant\^onio Carlos 6627, CEP 31270-901, Belo Horizonte - MG, Brazil
}

\email{vitors@cdtn.br}

% Optional disclaimer: remove this command to hide
%\disclaimer{Notice: this manuscript is a work of fiction. Any resemblance to
%actual articles, living or dead, is purely coincidental.}


%%%% Abstract
\begin{document}
\vspace*{-42pt}
\begin{strip}
\centering{\parbox{153mm}{{\bf Abstract} \itshape - 
This is the abstract one day I'll write
}\par}
\vspace*{14pt}
\end{strip}


%%%%%%%%%%%%%%%%%%%%%%%%%%%%%%%%%%%%%%%%%%%%%%%%%%%%%%%%%%%%%%%%%%%%%%%%%%%%%%%%
\section{Introduction}


Serpent Nuclear code is... (some paragraphs of Serpent).



%There are many consideratios to be taken in account for the process of implementing new
%features in an existing code. Specifically regarding the Serpent Nuclear code,
Before explaining the reason of implementing GPU capabilities to Serpent Nuclear code,
it must be said that Serpent implements two forms of CPU parallelization. In order to
do that, Serpent developers chose to make use of established technologies. The idea of using mature
(and in this case, free and open-source) libraries and technologies is to avoid re-inventing the wheel
while taking advantage of extensively tested implementations. The objective here is to
introduce two of the main concepts behind the technologies used by Serpent to provide parallelism.

The idea here is not to explain the intrinsics of each one, but to briefly present their basic
concepts, how the work and what the differences between them. All of this keeping in mind the
perspective of Serpent code development. The first one is OpenMP\cite{Dagum1998}, which is an API
(application programming interface) supporting shared memory multiprocessing programming. OpenMP consists
in a set of library routines, compiler directives and environment variables, and work in many different platforms.

%to execute parallel calculations starting many threads. Another one
OpenMPI\cite{openmpi2004} is one implementation of message passing interface API, to carry on parallel calculations on many processors.


\subsubsection{OpenMP}


\subsubsection{MPI}
%It is not the objective of this paper
%to explain the differences between OpenMP and OpenMPI, but it is enough to say that the former
%is thread based, which means processes (or threads) share the memory space while for the latter
%each process has its own memory space and information is exchanged by the use of standard messages.
%Both techniques are well-known and widely used as forms of implementing and run parallel applications.

That said, a question raises: why bother spending time implementing OpenCL parallelization
in a code already parallel capable? The answer is straightforward: to take
advantage of the processing power available in graphic processing units (GPU).
However, to fully answer the question considering aspects of general-purpose computing on graphics
processing units (GPGPU), it is necessary to give a small introduction on how GPU's work, what is OpenCL and why
it can be useful for Serpent.


\subsection{Graphic Processing Units}

Each GPU manufacturer provides its own set of compilers and libraries to allow general-purpose
programming of their video cards. This can be a problem from the point of view of generality of
the code being written. A code aimed to a specific hardware might go under deep re-factoring to
be able to run in a different hardware. This is where OpenCL comes in handy.

\subsection{What is OpenCL?}

OpenCL\cite{Stone2010} is a standard brought to life by a consortium of hardware and software
manufactures convinced by the idea that offering a standardized way of programming for different
hardware would be beneficial for both the users (in this case, developers) and the industry.

OpenCL brings to Serpent the ability to take advantage of a variety of different hardware
without hardware specific codification. This means, if a user has Graphic Processor Unities (GPU)
available, Serpent is able to use this additional computational power to simulate
more neutron histories.

%%%%%%%%%%%%%%%%%%%%%%%%%%%%%%%%%%%%%%%%%%%%%%%%%%%%%%%%%%%%%%%%%%%%%%%%%%%%%%%%
\section{Theory}

\subsection{Random Numbers Generation}
The absolutely key concept that must be understood in order to use the Monte Carlo method is
the generation of random numbers. Unsurprisingly, there are many well-known methods of varying
complexity and offering different quality\footnote{Quality is used in the sense of
  repeatability, robustness, speed and randomness of the numbers.} of the random numbers.
In this section the way the Serpent code generates its random numbers is explained together
with the implemented RNG method in OpenCL.


\subsection{Woodcock Delta-tracking}

One of the most important features of Serpent Nuclear Code is it's ability in combining
traditional surface tracking - MCNP? cite - for neutrons with the Woodcock delta-tracking method \cite{Woodcock1965}.

%%%%%%%%%%%%%%%%%%%%%%%%%%%%%%%%%%%%%%%%%%%%%%%%%%%%%%%%%%%%%%%%%%%%%%%%%%%%%%%%
\section{Previous work}

The processing power offered by GPUs is not neglected by scientists and software developers.
Bergmann\cite{Bergmann2014} developed a framework for continuous energy Monte Carlo neutron
transport (named WARP) with a careful design aimed to take advantages of the GPU architecture while, at the
same time, proposing new solutions and algorithms to avoid the penalties imposed by data transfer
or and memory traffic. Some of the ideas used in WARP were adapted to be implemented in Serpent in
the present work.

Another evolution on the implementation of Monte Carlo method for neutron transport from CPUs to GPUs
is presented by Okubo\cite{Okubo2016}. In this work, the authors analyze the delta-tracking method,
proposed by Woodcock\cite{Woodcock1965}, and implement new algorithms which avoid data race in
GPUs. The results show the effectiveness of the proposed implementation in GPU compared to the CPU
implementation.

% Is this really important?
%\subsection{Sample methods in Serpent}

\subsection{General view of Serpent}

There are some simple information regarding aspects of Serpent design which are
useful for more easily understand (and work on) its source code. The very first one is related to its
main header files. Serpent has roughly eleven header files in its source code. Despite all of them
being fundamental to compile and have Serpent running, two of these files have information which
is basic to understand Serpent data structures and how data is stored and managed in memory.

The file \texttt{header.h} keeps all header files which must be included to run Serpent.
It also defines macros for a huge number of constants related to the simulation aspects, like
physics and geometry, materials information, reaction rates and so on. Function prototypes, function
macros, global variables, constants and structures are also declared or defined in this file.

The \texttt{locations.h} header file holds the information of array indexes. In other words, a set of
definitions which are used to index a position in an array aimed to hold information about the simulation.
This is better explained by an example: the macro \texttt{DATA\_PTR\_HOSTNAME} has a numerical value
which is the index of an array where a pointer to the string containing the machine hostname is kept.

Another valuable piece of information is that every function used by Serpent
is implemented in a source code with the same name. This makes straightforward to access the needed
function implementation. For example, the function \texttt{InitData()} is implemented in the file
\texttt{initdata.c}.

\subsection{IMPORTANT}
Carefully describe the ideas in the cited papers: probably ideas from both sources can be applied to
the implementation. After describing (and deeply studying them), in the implementation section those
ideas must be detailed in how they're applied in Serpent. \textbf{This will be the core of this work}.

\subsection{Memory management in Serpent}

Monte Carlo simulations are huge memory consumers. This eager for memory requires a careful
management of memory resources. Serpent deals with this issue with its own set of memory
management primitives implemented using standard functions.

This scheme avoids common problems of memory leaks and dangling pointers while helps to
have a standard way of using resources.

\textbf{How Serpent manages memory?}
Why Serpent uses big arrays instead of scattered elements in memory?
Is \textbf{coalescence} the reason? Check with Jaakko.

There are two reasons for using this rather special management memory
structures and functions in Serpent. First of all, since Serpent is
capable of run in parallel using both threaded parallelism and
message passing communication, this memory management scheme is
used to guarantee a thread safe execution in the cases where
the implemented version of MPI does not provide one.

\textbf{Another reason for having such huge arrays memory variables is to
provide a degree of coalescence for data in threaded parallel implementations.
(Certeza disso? Chute forte...)}

% Na prática, não vem ao caso quais os arquivos que implementam as 
% funções. No máximo depois listar como curiosidade, não um itemize.
%
%Serpent has five source files which have implementations related
%to memory management. They are:
%\begin{itemize}
%\item freemem.c: 
%\item memcount.c:
%\item preallocmem.c: void PreallocMem(long, long)
%\item reallocmem.c: long ReallocMem(long, long). WDB, RES1 and ACE arrays are allocated in this file.
%\item mem.c:
%\end{itemize}

Serpent has several functions implemented to manage memory resources. They are implemented in different
source files for the sake of code organization. For the purpose of this text, the focus is on the functions
and the way they work instead of on the source files, which will be listed at the end of this section for
the sake of completeness.

The function \texttt{void FreeMem()} has a quite simple implementation (in file \texttt{fremem.c}). It checks for each data structure
defined in Serpent for memory storage if it is already allocated and, if yes, frees the memory using
the function \texttt{Mem()} for this purpose, also closing any output file related to MPI processes
in the case they are existent.

The function \texttt{long MemCount()} (implemented in file \texttt{memcount.c}) keeps track
of the size allocated for a variable since the last
call. In its implementation, a function called \texttt{CalculateBytes()} and implemented in the file
\texttt{calculatebytes.c}, updates for all main data structures their allocated memory in bytes.
After the calculation, the byte counter is updated for every main data structure and the difference
in bytes since the last call is returned. The \texttt{MemCount()} is used to keep track of the increase
in memory consumption, a useful feature when implementing a memory greed method like Monte Carlo.

  
% Explicar a Mem por último, já que todas as outras usam ela.

The implementation of the function \texttt{void *Mem(long, ...)} is far more complex than the one for
memory release. Starting from its signature, it is possible to see that it returns a pointer to void
and has a non-fixed number of parameters. The returning pointer is defined as a void pointer in order
to allow casting to other types as needed by the code invoking the \texttt{Mem()}
function. The return pointer holds the address of the allocated data.

This function has different operation modes depending on the arguments. Serpent uses a huge set of definitions for
multiple purposes. Some of these are related to memory operations and are used to define the behavior function \texttt{Mem()}
when passed as arguments. These operation ``modes'' are:

%This means that accordingly to the mode set - macros
%are previously defined in the main header file of- the function works in different ways. The modes of operation for function \texttt{Mem()} are:

\begin{itemize}
\item \texttt{MEM\_ALLOC}
\item \texttt{MEM\_REALLOC}
\item \texttt{MEM\_FREE}
\item \texttt{MEM\_ALLOW}
\item \texttt{MEM\_DENY}
\end{itemize} 

The allocation of memory is performed using the \texttt{MEM\_ALLOC} mode. This mode uses two arguments, the first
corresponds to the number of elements - or the block size - to be allocated, while the second defines the size of each element.
For memory re-allocation the mode \texttt{MEM\_REALLOC} is used. It is important to remark that this mode for memory re-allocation
has nothing to do with the implementation of the function \texttt{ReallocMem(long, long)} (which will be explained timely). The re-allocation mode
in function \texttt{Mem()} (mem.c) checks for the arguments since it needs the variable (roughly the pointer to the address which must
be re-sized) and the new size for it. If the new size is valid, the function reallocates memory for the variable and returns the pointer.

%It is worth noting that, differently from usual C memory management functions, Serpent memory management functions related
%to allocation and re-allocation of memory need
%
% Queria dizer aqui que a memória tem que estar definida antes (RDB), mas não é bem assim. Tenho que ter certeza.

The mode \texttt{MEM\_FREE} simply frees the memory space previously allocated.

The modes \texttt{MEM\_ALLOW} and \texttt{MEM\_DENY} work in the same way, but with opposite objectives. As
their names say, the former marks a memory space\footnote{Space in this context means a certain element of a huge array
of elements which Serpent uses instead of many scattered elements in memory.} as allocable while the latter marks it as non-available
for allocation.

Perhaps the most complex function in Serpent for memory management is the \texttt{ReallocMem(long, long)}.
The comments included in the source code clearly asserts that this function allocates memory for
three of the main arrays used for data storage. However, the mechanism of this management is not
completely intuitive and deserves some comments.

Firstly, this is a re-allocation function which is also responsible for the very first memory
allocation. This is done by checking if data allocation is valid for the chosen array.

The source files containing memory related functions are \texttt{freemem.c},
\texttt{memcount.c}, \texttt{preallocmem.c}, \texttt{reallocmem.c}, \texttt{mem.c}
and \texttt{freefinix.c}\footnote{Serpent has the function \texttt{FreeFinix()} which
  is used exclusively for multiphysics calculations using the FINIX code. This function
  is not mentioned in the text since it is not part of the basic memory management functions
  but an advanced feature.}.


\textbf{How GPU memory is used in this implementation?}

\subsection{Data initialization in Serpent}
Internally, Serpent starts by initializing its data structures inside a loop. This loop checks
whether more coefficients are still to be calculated. It continues until \begin{verbatim}
  ((long)RDB[DATA_COEF_CALC_SPECIAL_MODE]
      == SPECIAL_COEF_MODE_HIS_ONLY)
\end{verbatim}.

Inside the loop a conditional checks if the process rank is zero to perform many checks and initializations.
This is to guarantee that MPI processes does not perform tasks which are not supposed to be made in parallel.

It is impossible (and impractical) to describe all actions performed by Serpent in this text. However, some
procedures are key to understand how data is initialized, stored and managed in order to further explore
the parallelization options in graphical processing units.

\section{Profiling Serpent}

Serpent (version 2.1.29) was executed using a profiling tool, namely gprof [cite it] in order
to assess the time spent by functions which can give suggestion of code chunks candidates for
parallelization. The execution of a simple problem \textbf{without} burn-up show two functions:
\begin{verbatim}
void AddPrivateRes()
double MacroXS()
\end{verbatim}
However, considering the parallel nature of the Monte Carlo method, another option to be considered
for parallelization is to evaluate where the MPI and OpenMP calls are performed and add GPU code
in order to improve the load balance of the system using the GPU.

A first problem is to consider GPU plus MPI and then eventually consider other combinations.

\subsection{Data location in Serpent}

Data location relates to the very specific problem of having data used by a function or an
structure scattered in different locations in the physical memory. For the wide range of
applications, this is absolutely irrelevant due to the speed of data access in cache memories.

However, when dealing with a GPU, fetch data from one or another location can be the very
bottle-neck in the implementation of an algorithm.

\subsection{OpenCL in Serpent}

The way it works with CPUs.
The way it works with GPUs.

%%%%%%%%%%%%%%%%%%%%%%%%%%%%%%%%%%%%%%%%%%%%%%%%%%%%%%%%%%%%%%%%%%%%%%%%%%%%%%%%
% Comentado
\iffalse
\begin{subequations} \label{eqs:fullTransport}
\begin{multline} \label{eq:fullTransportVol}
  \vec{\Omega}\vd \grad \psi(\vec{x}, \vec{\Omega})
  + \sigma(\vec{x}) \psi (\vec{x}, \vec{\Omega})
\\ =
  \frac{\sigma_s(\vec{x})}{4\pi} \int_{4\pi} \psi(\vec{x},\vec{\Omega}')
  \ud\Omega' + \frac{q(\vec{x})}{4\pi}
  \equiv \frac{1}{4\pi} Q(\vec{x}) \,,
\end{multline}
inside $\vec{x} \in V$, $\vec{\Omega} \in 4\pi$, with an incident boundary
condition
\begin{equation} \label{eq:fullTransportBndy}
  \psi(\vec{x}, \vec{\Omega}) = \psi^b(\vec{x}, \vec{\Omega}) \,,
 \quad \vec{x} \in \partial V, \ \vec{\Omega} \vd \vec{n} < 0\,.
\end{equation}
\end{subequations}
\fi
%%%%%%%%%%%%%%%%%%%%%%%%%%%%%%%%%%%%%%%%%%%%%%%%%%%%%%%%%%%%%%%%%%%%%%%%%%%%%%%%

%%%%%%%%%%%%%%%%%%%%%%%%%%%%%%%%%%%%%%%%%%%%%%%%%%%%%%%%%%%%%%%%%%%%%%%%%%%%%%%%
\section{Results and Analysis}

In order to assess the results of using OpenCL (supposing the implementation works)
is to simulate the TRIGA IPR-R1 in different setups:
\begin{enumerate}
\item Serpent with OpenMP;
\item Serpent with MPI;
\item Serpent with hybrid OpenMP/MPI (is it possible?);
\item Serpent with OpenCL CPU;
\item Serpent with OpenCL GPU;
  \item Serpent with OpenCL CPU and GPU;
  \end{enumerate}

%%%%%%%%%%%%%%%%%%%%%%%%%%%%%%%%%%%%%%%%%%%%%%%%%%%%%%%%%%%%%%%%%%%%%%%%%%%%%%%%
\subsection{Subsection Goes Here}
The user must manually capitalize initial letters of a subsection heading.

%\begin{figure}[ht] % replace 't' with 'b' to force it to be on the bottom
%  \centering
%  \includegraphics{example_figure}
%  \caption{Captions are flush with the left.}
%  \label{fig:voltage}
%\end{figure}

\iffalse
Later on, we can include a table, even one that spans two columns such as
Table~\ref{tab:widetable}.
%%%%%%%%%%%%%%%%%%%%%%%%%%%%%%%%%%%%%%%%
\begin{table*}[htb]
  \centering
\begin{tabular}{llllllllll}\toprule
      & $\phi_T(0)$      & $\phi_T(10)$      & $\phi_T(20)$      &
      $\phi_D(0)$      & $\phi_D(10)$      & $\phi_D(20)$      & $\rho$      &
      $\varepsilon$      & $N_\text{it}$
\\ \midrule
$c=0.999$  & 0.9038 & 20.63 & 31.24 & 0.9087 & 20.63 & 31.23 & 0.2192 & $10^{-7}$ & 15
\\
$c=0.990$  & 0.3675 & 13.04 & 24.7 & 0.3696 & 13.04 & 24.69 & 0.2184 & $10^{-7}$ & 15
\\
$c=0.900$  & 0.009909 & 4.776 & 17.64 & 0.009984 & 4.786 & 17.63 & 0.2118 & $10^{-7}$ & 14
\\
$c=0.500$  & $6.069\times 10^{-5}$ & 2.212 & 15.53 & 6.213$\times 10^{-5}$ & 2.239 & 15.53 & 0.2068 & $10^{-7}$ & 13
\\
\bottomrule
\end{tabular}
  \caption{This is an example of a really wide table which might not normally
  fit in the document.}
  \label{tab:widetable}
\end{table*}
%%%%%%%%%%%%%%%%%%%%%%%%%%%%%%%%%%%%%%%%
Notice how the table reference uses a Roman numeral
for its numbering scheme, whereas the figure reference uses an Arabic numeral.
For one-column tables, use the \verb|table| environment; two-column tables use
\verb|table*|. The same applies to figures.
\fi

%%%%%%%%%%%%%%%%%%%%%%%%%%%%%%%%%%%%%%%%%%%%%%%%%%%%%%%%%%%%%%%%%%%%%%%%%%%%%%%%
\subsection{Another Subsection}
Excessive sectioning in a three-page document is discouraged, but here are more
subsections to demonstrate compliance with the ANS formatting guidelines.

\subsubsection{Third-level Heading}
This subsubsection shows compliance with the ANS-specified standard. This level
of heading should be used rarely.

\subsubsection{Another Such Heading}
And, if you really think you need a third-level heading, you should make sure
that your subsection needs at least two of them.

%%%%%%%%%%%%%%%%%%%%%%%%%%%%%%%%%%%%%%%%%%%%%%%%%%%%%%%%%%%%%%%%%%%%%%%%%%%%%%%%
\section{Conclusions}

The thesis of Wen \cite{Wen2017} verses about heterogeneous multi-tasking, while
the master dissertation of Jespersen \cite{Jespersen2015} uses OpenCL to implement
a Monte Carlo method of stock options. The wide cited work of Owens \cite{Owens2007}
is a bit dated, but is a good resource explaining the GPU concepts. Two yet non-read articles
are the one of Liang \cite{Liang2017} about Monte Carlo simulation using GPU and this of Dimarco
\cite{Dimarco2017} which is related to the Boltzmann equations (these two are from the Journal
of Computational Physics).

%%%%%%%%%%%%%%%%%%%%%%%%%%%%%%%%%%%%%%%%%%%%%%%%%%%%%%%%%%%%%%%%%%%%%%%%%%%%%%%%
\appendix
\section{Appendix}

Numbering in the appendix is different:
\begin{equation} \label{eq:appendix}
  2 + 2 = 5\,.
\end{equation}
and another equation:
\begin{equation} \label{eq:appendix2}
  a + b = c\,.
\end{equation}

%%%%%%%%%%%%%%%%%%%%%%%%%%%%%%%%%%%%%%%%%%%%%%%%%%%%%%%%%%%%%%%%%%%%%%%%%%%%%%%%
\section{Acknowledgments}
This material is based upon work supported a Department of Energy Nuclear
Energy University Programs Graduate Fellowship.

%%%%%%%%%%%%%%%%%%%%%%%%%%%%%%%%%%%%%%%%%%%%%%%%%%%%%%%%%%%%%%%%%%%%%%%%%%%%%%%%
\section{Disclaimer}
The main author is licensed to use and modify Serpent source-code for research
and local purposes, not being allowed to share any changes without the formal
agreement of Serpent copyright holders.

%%%%%%%%%%%%%%%%%%%%%%%%%%%%%%%%%%%%%%%%%%%%%%%%%%%%%%%%%%%%%%%%%%%%%%%%%%%%%%%%
\bibliographystyle{ans} % Don't forget to run BibTeX !
\bibliography{serpent-gpu}

\end{document}
