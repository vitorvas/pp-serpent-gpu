\documentclass[twoside,a4paper,12pt,english,draft]{anstrans}

%%%% packages and definitions (optional)
\usepackage{graphicx} % allows inclusion of graphics
\usepackage{booktabs} % nice rules (thick lines) for tables
\usepackage{microtype} % improves typography for PDF


\newcommand{\SN}{S$_N$}
\renewcommand{\vec}[1]{\bm{#1}} % vector is bold italic
\newcommand{\vd}{\bm{\cdot}} % slightly bold vector dot
\newcommand{\grad}{\vec{\nabla}} % gradient
\newcommand{\ud}{\mathop{}\!\mathrm{d}} % upright derivative symbol


%%%% changes from the original 'anstrans' class
\usepackage{fancyhdr} % allows headers and footers

\renewcommand\headrule{} % remove underline in the header
\setcounter{secnumdepth}{1}
\renewcommand{\thesection}{\Roman{section}.}
\renewcommand{\thesubsection}{\arabic{subsection}.}
\renewcommand{\thesubsubsection}{\Alph{subsubsection}.}
\makeatletter
\renewcommand*{\@seccntformat}[1]{\csname the#1\endcsname\hspace{1mm}}
\makeatother


%%%% Header
%\pagestyle{fancy}
%\fancyhf{}
%\fancyhead[L]{\fontsize{9}{9} \itshape
%M\&C 2017 - International Conference on Mathematics \& Computational Methods Applied to Nuclear Science \& Engineering,
%\\ Jeju, Korea, April 16-20, 2017, on USB (2017)
%}


%%%% Maketitle
\title{Serpent nuclear code OpenCL extension}
\author{Vitor V. A. Silva,$^{*}$}

\institute{
$^{*}$Centro de Desenvolvimento da Tecnologia Nuclear, CEP: 30......
Belo Horizonte - MG, Brazil
}

\email{vitors@cdtn.br}

% Optional disclaimer: remove this command to hide
%\disclaimer{Notice: this manuscript is a work of fiction. Any resemblance to
%actual articles, living or dead, is purely coincidental.}


%%%% Abstract
\begin{document}
\vspace*{-42pt}
\begin{strip}
\centering{\parbox{153mm}{{\bf Abstract} \itshape - 
This is the abstract one day I'll write
}\par}
\vspace*{14pt}
\end{strip}


%%%%%%%%%%%%%%%%%%%%%%%%%%%%%%%%%%%%%%%%%%%%%%%%%%%%%%%%%%%%%%%%%%%%%%%%%%%%%%%%
\section{Introduction}

Serpent Nuclear code is... (some paragraphs of Serpent).

There are many consideratios to be taken in account during the process of implementing new
features in an existing code. Specifically regarding the Serpent Nuclear code, it must be
said that it already implements two forms of CPU parallelization. Serpent is able to use
OpenMP\cite{omp} to execute threads calculations. It is also capable of using OpenMPI\cite{openmpi}
to carry on parallel calculations on many processors. It is not the objective in this paper
to explain the differences between OpenMP and OpenMPI, but it is enough to say that the former
is thread based, which means processes (or threads) share the memory space while for the latter
each process has its own memory space and information is exchanged by the use of messages.

That said, a question raises: why bother spending time in implementing OpenCL parallelization
in a code already parallel capable. The answer is more or less straighfoward. However, to fully
answer the question, it is necessary to give a small introduction on what is OpenCL and why
it can be usefull for Serpent.

\subsection{What is OpenCL?}

OpenCL\cite{opencl} is a...

OpenCL brings to Serpent the ability to take advantage of a variety of different hardware
without special codification. This means, if a user has Graphic Processor Unities (GPU)
available, Serpent is able to use this additional computational power to simulate
more neutron histories.

%%%%%%%%%%%%%%%%%%%%%%%%%%%%%%%%%%%%%%%%%%%%%%%%%%%%%%%%%%%%%%%%%%%%%%%%%%%%%%%%
\section{Theory}

\subsection{Random Numbers Generation}
The absolutely key concept that must be understood in order to use the Monte Carlo method is
the generation of random numbers. Unsurprisenly, there are many well-known methods of varying
complexity and offering different quality\footnote{Quality is used in the sense of
  repeatbility, robustness, speed and randonmness of the numbers.} of the random numbers.
In this section the way the Serpent code generates its random numbers is explained.

\subsection{Sample methods in Serpent}

\subsection{Memory management in Serpent}

Monte Carlo simulations are huge memory consumers. This eager for memory requires a careful
management of memory resources. Serpent deals with this issue with its own set of memory
management primitives implemented using standard functions.

This scheme avoids common problems of memory leaks and dangling pointers while helps to
have a standard way of using resources.

\textbf{How Serpent manages memory?}

\subsection{Profiling Serpent}

Serpent (version 2.1.29) was executed using a profiling tool, namely gprof [cite it] in order
to assess the time spent by functions which can give suggestion of code chunks candidates for
parallelization. The execution of a simple problem without burn-up show two functions:
\begin{verbatim}
void AddPrivateRes()
double MacroXS()
\end{verbatim}
However, considering the parallel nature of the Monte Carlo method, another option to be considered
for parallelization is to evaluate where the MPI and OpenMP calls are performed and add GPU code
in order to improve the load balance of the system using the GPU.

A first problem is to consider GPU plus MPI and then eventually consider other combinations.

\subsection{Data location in Serpent}

Data location relates to the very specific problem of having data used by a function or an
structure scattered in different locations in the physical memory. For the wide range of
applications, this is absolutely irrelevant due to the speed of data access in cache memories.

However, when dealing with a GPU, fetch data from one or another location can be the very
bottle-neck in the implementation of an algorithm.

\subsection{OpenCL in Serpent}

The way it works with CPUs.
The way it works wiht GPUs.

%%%%%%%%%%%%%%%%%%%%%%%%%%%%%%%%%%%%%%%%%%%%%%%%%%%%%%%%%%%%%%%%%%%%%%%%%%%%%%%%
% Comentado
\iffalse
\begin{subequations} \label{eqs:fullTransport}
\begin{multline} \label{eq:fullTransportVol}
  \vec{\Omega}\vd \grad \psi(\vec{x}, \vec{\Omega})
  + \sigma(\vec{x}) \psi (\vec{x}, \vec{\Omega})
\\ =
  \frac{\sigma_s(\vec{x})}{4\pi} \int_{4\pi} \psi(\vec{x},\vec{\Omega}')
  \ud\Omega' + \frac{q(\vec{x})}{4\pi}
  \equiv \frac{1}{4\pi} Q(\vec{x}) \,,
\end{multline}
inside $\vec{x} \in V$, $\vec{\Omega} \in 4\pi$, with an incident boundary
condition
\begin{equation} \label{eq:fullTransportBndy}
  \psi(\vec{x}, \vec{\Omega}) = \psi^b(\vec{x}, \vec{\Omega}) \,,
 \quad \vec{x} \in \partial V, \ \vec{\Omega} \vd \vec{n} < 0\,.
\end{equation}
\end{subequations}
\fi
%%%%%%%%%%%%%%%%%%%%%%%%%%%%%%%%%%%%%%%%%%%%%%%%%%%%%%%%%%%%%%%%%%%%%%%%%%%%%%%%

%%%%%%%%%%%%%%%%%%%%%%%%%%%%%%%%%%%%%%%%%%%%%%%%%%%%%%%%%%%%%%%%%%%%%%%%%%%%%%%%
\section{Results and Analysis}

In order to assess the results of using OpenCL (supposing the implementation works)
is to simulate the TRIGA IPR-R1 in different setups:
\begin{enumerate}
\item Serpent with OpenMP;
\item Serpent with MPI;
\item Serpent with hybrid OpenMP/MPI (is it possible?);
\item Serpent with OpenCL CPU;
\item Serpent with OpenCL GPU;
  \item Serpent with OpenCL CPU and GPU;
  \end{enumerate}

%%%%%%%%%%%%%%%%%%%%%%%%%%%%%%%%%%%%%%%%%%%%%%%%%%%%%%%%%%%%%%%%%%%%%%%%%%%%%%%%
\subsection{Subsection Goes Here}
The user must manually capitalize initial letters of a subsection heading.

%\begin{figure}[ht] % replace 't' with 'b' to force it to be on the bottom
%  \centering
%  \includegraphics{example_figure}
%  \caption{Captions are flush with the left.}
%  \label{fig:voltage}
%\end{figure}

\iffalse
Later on, we can include a table, even one that spans two columns such as
Table~\ref{tab:widetable}.
%%%%%%%%%%%%%%%%%%%%%%%%%%%%%%%%%%%%%%%%
\begin{table*}[htb]
  \centering
\begin{tabular}{llllllllll}\toprule
      & $\phi_T(0)$      & $\phi_T(10)$      & $\phi_T(20)$      &
      $\phi_D(0)$      & $\phi_D(10)$      & $\phi_D(20)$      & $\rho$      &
      $\varepsilon$      & $N_\text{it}$
\\ \midrule
$c=0.999$  & 0.9038 & 20.63 & 31.24 & 0.9087 & 20.63 & 31.23 & 0.2192 & $10^{-7}$ & 15
\\
$c=0.990$  & 0.3675 & 13.04 & 24.7 & 0.3696 & 13.04 & 24.69 & 0.2184 & $10^{-7}$ & 15
\\
$c=0.900$  & 0.009909 & 4.776 & 17.64 & 0.009984 & 4.786 & 17.63 & 0.2118 & $10^{-7}$ & 14
\\
$c=0.500$  & $6.069\times 10^{-5}$ & 2.212 & 15.53 & 6.213$\times 10^{-5}$ & 2.239 & 15.53 & 0.2068 & $10^{-7}$ & 13
\\
\bottomrule
\end{tabular}
  \caption{This is an example of a really wide table which might not normally
  fit in the document.}
  \label{tab:widetable}
\end{table*}
%%%%%%%%%%%%%%%%%%%%%%%%%%%%%%%%%%%%%%%%
Notice how the table reference uses a Roman numeral
for its numbering scheme, whereas the figure reference uses an Arabic numeral.
For one-column tables, use the \verb|table| environment; two-column tables use
\verb|table*|. The same applies to figures.
\fi

%%%%%%%%%%%%%%%%%%%%%%%%%%%%%%%%%%%%%%%%%%%%%%%%%%%%%%%%%%%%%%%%%%%%%%%%%%%%%%%%
\subsection{Another Subsection}
Excessive sectioning in a three-page document is discouraged, but here are more
subsections to demonstrate compliance with the ANS formatting guidelines.

\subsubsection{Third-level Heading}
This subsubsection shows compliance with the ANS-specified standard. This level
of heading should be used rarely.

\subsubsection{Another Such Heading}
And, if you really think you need a third-level heading, you should make sure
that your subsection needs at least two of them.

%%%%%%%%%%%%%%%%%%%%%%%%%%%%%%%%%%%%%%%%%%%%%%%%%%%%%%%%%%%%%%%%%%%%%%%%%%%%%%%%
\section{Conclusions}

The thesis of Wen \cite{Wen2017} verses about heterogeneous multi-tasking, while
the master dissertation of Jespersen \cite{Jespersen2015} uses OpenCL to implement
a Monte Carlo method of stock options. The wide cited work of Owens \cite{Owens2007}
is a bit dated, but is a good resource explaining the GPU concepts. Two yet non-read articles
are the one of Liang \cite{Liang2017} about Monte Carlo simulation using GPU and this of Dimarco
\cite{Dimarco2017} which is related to the Boltzmann equations (these two are from the Journal
of Computational Physics).

%%%%%%%%%%%%%%%%%%%%%%%%%%%%%%%%%%%%%%%%%%%%%%%%%%%%%%%%%%%%%%%%%%%%%%%%%%%%%%%%
\appendix
\section{Appendix}

Numbering in the appendix is different:
\begin{equation} \label{eq:appendix}
  2 + 2 = 5\,.
\end{equation}
and another equation:
\begin{equation} \label{eq:appendix2}
  a + b = c\,.
\end{equation}

%%%%%%%%%%%%%%%%%%%%%%%%%%%%%%%%%%%%%%%%%%%%%%%%%%%%%%%%%%%%%%%%%%%%%%%%%%%%%%%%
\section{Acknowledgments}
This material is based upon work supported a Department of Energy Nuclear
Energy University Programs Graduate Fellowship.

%%%%%%%%%%%%%%%%%%%%%%%%%%%%%%%%%%%%%%%%%%%%%%%%%%%%%%%%%%%%%%%%%%%%%%%%%%%%%%%%
\bibliographystyle{ans} % Don't forget to run BibTeX !
\bibliography{serpent-gpu}

\end{document}
